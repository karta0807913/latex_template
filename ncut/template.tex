%!TEX program = xelatex
% https://minsky.pixnet.net/blog/post/26804386
% https://blog.csdn.net/l_changyun/article/details/86716758

% \documentclass[journal]{article} %普通的
% \documentclass[journal]{IEEEtran} % ieee 格式
\documentclass[12pt]{ncut}

%%-------------------------packages-----------------------------------%%
\usepackage{graphics}% Include figure files
\usepackage{dcolumn}% Align table columns on decimal point
\usepackage{amsmath}

\usepackage{algorithm}
\usepackage{algpseudocode}
\usepackage{epsfig}

%%--------------------------------------------------------------------%%

\begin{document}
\pagenumbering{roman}
\begin{titlepage}
  \renewcommand{\baselinestretch}{1.2}
    \begin{center}
        \vspace{0cm}
        \Huge
        \textbf{國立勤益科技大學\\
        冷凍空調與能源系碩士班\\
        碩士學位論文}

        \vspace{2cm}
        \Huge
        \bfseries
        便利商店環境下使用行動支付之\\
        消費者行為研究\\
        -以連鎖便利商店A超商為例\\
        Optional English TitleEnglish Title Example
        Optional English Title

        \vspace{1.5cm}

        \Large
        研究生:王小明
        \vspace{1.5cm}

        指導教授:李大仁 博士

        \vfill

        中華民國一百零九年一月

    \end{center}
    \newpage
\end{titlepage}
\begin{titlepage}
    \begin{center} % 審定書
      \fontsize{22pt}{22pt}
      \textbf{國立勤益科技大學\\
        冷凍空調與能源系碩士班\\
        論文口試委員會審定書}\\
    \end{center}
    \vspace{1cm}
    \large
    \begin{center}
      本校\underline{冷凍空調與能源系}碩士班\underline{王小明}君
    \end{center}

    \noindent 所提論文\underline{便利商店環境下使用行動支付之消費者行為研究-以連鎖便利商}\\
    \underline{店A超商為例}

    \vspace{2cm}
    \noindent 合於碩士資格水準,業經本委員會評審認可。

    \vspace{1cm}
    \noindent 口試委員:
    \vfill

    \noindent 指導教授:

    \makebox[0pt]{}\newline % for empty line
    系(所)主管:

    \makebox[0pt]{}\newline % for empty line
    中華民國  年 月
    \newpage
\end{titlepage}


\begin{center} %正文
    \Large
    \textbf{便利商店環境下使用行動支付之消費者行為研究\\
    -以連鎖便利商店A超商為例}
\end{center}

\section{摘要}
\begin{center}
    (略)
\end{center}
\noindent 關鍵字詞:

\newpage

\section{ABSTRACT}

\textbf{Start writing abstract from here. Start writing abstract from here. Start writing abstract from here. Start writing abstract from here. Start writing abstract from here. Start writing abstract from here. Start writing abstract from here. Start writing abstract from here.\\
Keywords:}

\newpage

\section{誌謝}

\indent
所有對於研究提供協助之人或機構,作者都可在誌謝中表達感謝之意。標題使用 20pt粗標楷體,並於上、下方各空一行(1.5倍行高字型12pt空行)後鍵入內容。致謝頁須編頁碼(小寫羅馬數字表示頁碼)。

\newpage

\tableofcontents
\newpage

\listoftables
\newpage

\listoffigures
\newpage

\pagenumbering{arabic}

\setcounter{section}{0} %重新編號
\section{第一章緒論}
blablabla

\subsection{第一層子標題}
blablabla

\subsubsection{第二層子標題}
blablabla

\newpage

\section{第二章章節標題太長可以排列成兩行之範例}
blablabla
\subsection{第一層子標題}
blablabla
\subsubsection{第二層子標題}
blablabla

\newpage

\newpage
%% \addcontentsline{toc}{section}{\refname}
%% \renewcommand{\bibname}{參考文獻}
\begin{thebibliography}{9}
\bibitem{test}
  Doak, E. M. (2002). Decisions, Decisions, Decisions: A Tale of Special Collections in the Small Academic Library. Acquisitions Librarian, 14(27), 41. 
\bibitem{testa}
Horava, T. (2017). Dollars and Decision-Making: What is a "Collection" Nowadays? Technicalities, 37(6), 14-17. 
\bibitem{testb}
Lukes, R., Markgren, S., \& Thorpe, A. (2016). E-Book Collection Development: Formalizing a Policy for Smaller Libraries. Serials Librarian, 70(1-4), 106-115.
\bibitem{testc}
Rogers, J. P., \& Wesley, K. (2015). Reaching New Horizons: Gathering the Resources Librarians Need to Make Hard Decisions. Serials Librarian, 68(1-4), 64-77. doi:10.1080/0361526X.2015.1016831
\bibitem{testd}
方文昌、張重昭、林建煌、汪志堅(2006)。台灣行銷與消費者行為研究論文在國際頂尖期刊的發表現況。行銷評論,3(2),137-148。doi:10.29931/mr.200606.0001
\bibitem{teste}
吳杰倫(2019)。服務品質、顧客滿意度與忠誠度之關係─以R珠寶公司為例。國立臺北科技大學管理學院EMBA泰國專班,臺北市。
\bibitem{testf}
杜健、賈俊秀(2016)。社交媒體對消費者行為影響下的企業最佳產能策略。「中國系統工程學會第19屆學術年會」發表之論文,北京。
\bibitem{testg}
黃明居、吳東陽(2014)。圖書館電子資源之加值應用以引用文獻分析系統之建置為例,圖書館學與資訊科學,40,66-81。
\bibitem{testh}
蕭富峰(2009)。消費者行為(初版)。臺北市:智勝文化。
\end{thebibliography}

\end{document}