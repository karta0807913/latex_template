%!TEX program = xelatex
\documentclass[journal]{article} %普通的
% \documentclass[journal]{IEEEtran} % ieee 格式

\usepackage{graphicx}% Include figure files
\usepackage{dcolumn}% Align table columns on decimal point
\usepackage{bm}% bold math
\usepackage[UTF8]{ctex}
\usepackage{amsmath}
\usepackage{tabularx}

\usepackage{xeCJK}
\setmainfont{Times New Roman}[
  Path = ./fonts/english/,
  Extension = .ttf ,
  BoldFont = *-Bold ,
  ItalicFont = *-Italic ,
  BoldItalicFont = *-BoldItalic, ]
\setCJKmainfont{BiauKai}[
  Path = ./fonts/chinese/ ,
  Extension = .ttf ,
  BoldFont = Kaiti-Black, ]

\usepackage{algorithm}
\usepackage{algpseudocode}
\usepackage{graphics}
\usepackage{epsfig}
\usepackage{caption}
\usepackage[utf8]{inputenc}
\usepackage{hyperref}

\usepackage{graphicx}

\usepackage{biblatex} % bibertex
\addbibresource{chinese.bib}

\begin{document}

% caption figure 的prefix
% \begin{figure}
%     \includegraphics[width=\linewidth]{flow.png}
%     \caption{交易流程}
%     \label{flow_image} % 圖 x
% \end{figure}
\captionsetup[figure]{name=圖}

\title{標題}
\author{作者}
\date{\today}
\maketitle

% 目錄
\renewcommand\contentsname{目錄}
\tableofcontents

% 圖目錄
% \renewcommand\listfigurename{圖目錄}
% \listoffigures

% 表目錄
% \renewcommand\listtablename{表目錄}
% \listoftables

\newpage

%table
% \begin{table}[h!]
% \centering
% \begin{tabularx}{\linewidth}{l X} 
%  \hline
%  \textbf{Notation} & \textbf{Description}\\
%  \hline
%  A & a \\
%  B & b \\
%  \hline
% \end{tabularx}
% \caption{Table to test captions and labels}
% \label{table:1}
% \end{table}

% 演算法block中的Procedure, Function 區塊
% \algdef{SE}[PROCEDURE]{Procedure}{EndProcedure}%
% [2]{\algorithmicprocedure\ \textproc{#1}\ifthenelse{\equal{#2}{}}{}{(#2)}}%
% {\algorithmicend\ \algorithmicprocedure}%
% \algdef{SE}[FUNCTION]{Function}{EndFunction}%
% [2]{\algorithmicfunction\ \textproc{#1}\ifthenelse{\equal{#2}{}}{}{(#2)}}%
% {\algorithmicend\ \algorithmicfunction}%

% 演算法block
% \begin{algorithm}[htb]
%     \begin{algorithmic}[1]
%     \end{algorithmic}
% \end{algorithm}

% START HERE

測試用例 \cite{8643999}

% 參考文獻
% \newpage
% \addcontentsline{toc}{section}{\refname} %參考文件加入目錄
% \begin{thebibliography}{9}
% \bibitem{test}
%   description
% \end{thebibliography}

\newpage
\printbibliography[title={\centering 參考文獻}]

\end{document}
